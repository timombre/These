% la ligne ci-dessous est � ins�rer obligatoirement dans le pr�ambule du document avant \begin{document}

%\usepackage[a4paper]{meta-donnees}


% les lignes en bas sont � ins�rer obligatoirement apr�s \begin{document}

%%%%%%%%%%%%%%%%%%%%%%%%%%%%%%%%%%%%%%%%%%%%%%%%%%%%%%
%%             Commandes Meta-donn�es               %%
%%   � renseigner par les auteurs pour g�n�rer      %%
%%     la couverture mod�le Univ. Grenoble          %%
%%%%%%%%%%%%%%%%%%%%%%%%%%%%%%%%%%%%%%%%%%%%%%%%%%%%%%
%%      Fichier encod� au format ISO-8859-16        %%

%\Sethpageshift{???mm}   %%optionnel : � d�commenter si besoin pour ajout d'espace afin de center la couv�rture horizontalement (valeur par d�faut est -5.5mm)
%\Setvpageshift{???mm}   %%optionnel : � d�commenter si besoin pour ajout d'espace afin de center la couv�rture verticalement (valeur par d�faut est -15.5mm)


%\Universite{}    %%optionnel : � d�commenter et � renseigenr si vous voulez changer le non d'universit�
%\Grade{}         %%optionnel : � d�commenter et � renseigenr si vous voulez changer le grade
\Specialite{Physique Th\'eorique}
\Arrete{7 ao\^{u}t 2006}
\Auteur{Timoth\'ee MENAIS}
\Directeur{Arnaud BUHOT}
\CoDirecteur{Stefano MOSSA}    %%optionnel : � d�commenter et � renseigenr si pr�sence d'un Co-directeur de th�se
\Laboratoire{CEA Grenoble\\
Institut Nanosciences et Cryog\'enie (INAC)\\
\footnotesize{Service: Structures et Propri\'et\'es d'Architectures Mol\'eculaires (SPRaM)
\\
Groupe: Chimie pour la Reconnaissance et l'Etude des Assemblages\\
 Biologiques (CREAB)}}
\EcoleDoctorale{l'\'Ecole doctorale de Physique}         
\Titre{Etude th\'eorique de la translocation de
  biomol\'ecules \`a travers une membrane fine}
%\Soustitre{}      %%optionnel : � d�commenter et � renseigenr si pr�sence d'un sous-titre de th�se
\Depot{21 janvier 2016}       




% Commande pour cr�ation de nouvelles cat�gories dans le jury:

%\UGTNewJuryCategory{...NomDeLaCategorie...}{...Definition...}

% Exemple \UGTNewJuryCategory{UGTFamille}{Membre de la famille} que nous ajoutons dans la commande \Jury ci-dessous sous la forme \UGTFamille{Jean Rousseau}{(...titre_et_affiliation...s'il_y_en_a...)}


\Jury{
\UGTPresident{M.Bertrand FOURCADE}{Professeur \`a l'universit\'e Joseph Fourrier (Grenoble)}


\UGTRapporteur{M.Nicolas DESTAINVILLE}{Professeur \`a l'universit\'e Paul Sabatier (Toulouse)}  %% 1er rapporteur
\UGTRapporteur{M.Enrico CARLON}{Professeur \`a l'universit\'e Ku Leuven (Leuven, Belgique)}  %% second rapporteur
 
\UGTExaminateur{M.Andrea PARMEGGIANI}{Directeur de recherche \`a l'universit\'e Montpellier II}     %% 1er examinateur
 %% 3�me examinateur

\UGTDirecteur{M.Arnaud BUHOT}{Ing\'enieur de recherche au CEA Grenoble}      %% Directeur de th�se
\UGTCoDirecteur{M.Stefano MOSSA}{Ing\'enieur de recherche au CEA Grenoble}     %% Co-Directeur de th�se s'il y en a

}

\MakeUGthesePDG    %% tr�s important pour g�n�rer la couv�rture de th�se

