\thispagestyle{empty}

\newpage
\thispagestyle{empty}

\section*{Résumé}

\noindent Afin de faciliter le séquençage des génomes, une troisième génération de systèmes de séquençage est nécessaire. La translocation de biomolécules est le phénomène clé mis en jeu dans l'optique d'un séquençage par nanopore. Nous nous sommes intéressé à ce phénomène avec une approche à la fois théorique et numérique. Dans  le cadre de nos travaux, nous avons tout d'abord élaboré un modèle dit gros grain de polymère structuré (présentant une structure proche de celle de l'ADN) adapté à une étude par dynamique moléculaire. Après avoir vérifié la pertinence de notre modèle avec les lois de la physique statistique des polymères, nous nous sommes concentrés sur la translocation. Nous avons revisité le cas standard d'un nanopore au sein d'une membrane fixe et proposé un modèle théorique dans le cas de la traction du polymère. L'arrivée des membranes fines, clés du succès d'un éventuel séquençage non destructif par nanopore entraine de nouvelles intéractions avec la membrane qui n'ont pas encore été étudiées. Nous présentons la première étude numérique de grande ampleur sur ses intéractions. Nos résultats permettent d'étudier l'influence des vibrations, de la déformabilité et de la flexibilité de la membrane.\\




\noindent\textbf{Mots-clés:} \textit{physique statistique, translocation de polymères, étude numérique, biophysique, théorie, dynamique moléculaire, modèle gros grain}




\section*{Abstract}

\noindent So as to facilitate genomic sequencing, a third generation of sequencing devices is needed. Biopolymer translocation is the key phenomenon involved in nanopore sequencing prospects. We investigated this phenomenon through both theoretical and numerical approaches. Our work started with devicing a coarse grained structured polymer (with a DNA like structure) adapted to a molecular dynamics study. Once we have verified the reliability of our model towards statistical polymer physics, we focused on translocation. We reinvestigated the common case of a nanopore within a fix membrane and proposed a theoretical model for translocation under a pulling force. The arrival of thin membranes is key to an eventual success of non destructive sequencing with a nanopore. This enables new interactions with the membrane which have not been investigated yet. We provide the first large scale numerical investigation of such interactions. our results provide an insight over the influence of membrane vibrations, deformability and flexibility.\\




\noindent\textbf{Keywords:} \textit{statistical physics, polymer translocation, numerical study, biophysics, theory, molecular dynamics, coarse grain model}