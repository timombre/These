\section*{Remerciements}

Tout plein de gens à remercier
arnaud stefano veera iannis tout le créab camille zoéline :)
\textcolor{red}{test}

\newpage

\section*{Introduction}

Ce manuscrit de thèse présente les travaux réalisés au sein du groupe CREAB (Chimie pour la Reconnaissance et l’Etude d’Assemblages Biologiques) et en coopération avec le groupe PCI (Polymères Conducteurs Ioniques), deux groupes du laboratoire SPrAM (Structures et Propriétés d'Architectures Moléculaires) de l'INAC (Institut Nanosciences et Cryogénie) au sein du CEA (Commissariat à l'Energie Atomique et aux énergies alternatives) de Grenoble de Novembre 2012 à Novembre 2015.

Le sujet s'intègre dans le cadre de la translocation de biomolécules à travers une membrane fine, thème porté par l'équipe CREAB et étudié avec des outils numériques de dynamique moléculaire, compétences apportées par la collaboration avec le groupe PCI. \\

\textbf{La translocation de polymères} (notamment l 'ADN), est le passage d'un c\^{o}té à l'autre d'une membrane en traversant un pore situé dans cette dernière. L'étude de ce domaine est très active tant sur le plan expérimental que théorique. Ceci est principalement d\^{u} aux applications potentielles en biotechnologies et en médecine, notamment pour le séquençage du génome. En plus des potentielles applications techniques, ce thème est porteur de questions fondamentales très intéressantes d'un point de vu physique statistique des polymères.\\

\textbf{La dynamique moléculaire} est une technique de simulation numérique que nous avons appliquée au problème de la translocation. Il s'agit de définir les potentiels d’interactions entre les différents sous-systèmes (ou grains) du problème et d'intégrer les équations du mouvement pour obtenir une trajectoire. Cette méthode permet d'envisager des modèles de toutes sortes de complexité, en découpant le système en grains de nature variée, du simple atome à la molécule complète.\\

Avec ces deux thèmes en t\^{e}te, nous avons développé un modèle original de polymère adapté à l'étude statistique de la translocation de cha\^{i}nes structurées (telle l'ADN) à travers des membranes fines. Ce modèle, suffisamment simple pour obtenir des résultats assez nombreux pour \^{e}tre statistiquement significatifs nous a permis de sonder les effets de l'utilisation de membranes fines sur le phénomène de translocation, que ce soit via la vibration et la déformabilité du nano-pore ou bien encore la flexibilité de la membrane.\\

Ce manuscrit est structuré de la manière suivante:\\

Un \hyperref[intro]{premier chapitre} introductif présente le contexte technologique du séquençage de l'ADN par nanopores ainsi que la variété de modèles théoriques et numériques pertinents pour aborder la translocation de polymères. Dans le \hyperref[polymerevalidation]{second chapitre}, nous détaillons l'élaboration de notre modèle gros grain de polymère structuré et vérifions que ses propriétés sont accords avec la théorie de la physique des polymères. Le \hyperref[translocmurfixe]{troisième chapitre} présente les théories sur la translocation et les résultats obtenus avec notre modèle dans le cas d'une membrane fixe. Nos résultats numériques obtenus servent de base à un modèle théorique que nous proposons pour la translocation d'un polymère tiré par une de ses extrémités et de référence pour le \hyperref[effetsmembrane]{dernier chapitre} dans lequel nous étudions les effets des propriétés de la membrane sur la translocation, d'abord en se concentrant sur ses vibrations puis sur sa flexibilité.



